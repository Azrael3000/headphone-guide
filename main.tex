\documentclass{article}
\usepackage{url}
\usepackage[utf8]{inputenc}
\begin{document}
\title{Let's build electrostatic headphones}
\author{Arno Mayrhofer}
\maketitle

\tableofcontents

\newpage

\section{Introduction}
\label{s:intro}
The following document will describe the journey on how to construct your own electrostatic headphones starting with zero, zilch, nada, niente and nichts. In Section \ref{s:tools} we will list the tools required for the building process, which will be followed by a list of materials in Section \ref{s:materials}. The actual construction process is split into five parts, the building of the driver (Section \ref{s:driver}), enclosure (Section \ref{s:driver}), headband (Section \ref{s:headband}) and earpads (Section \ref{s:pads}) with a final assembly in Section \ref{s:assembly}. Finally, the last two chapters will deal with measurements (Section \ref{s:measurements}) and things that we would do differently for the second pair (Section \ref{s:future}).

The whole construction is based on the Head-Fi.org thread \cite{head-fi-diy-thread}, with different and more detailed sources (e.g. \cite{electrostatic-hp-design}, \cite{tcengineering-electrostatic-drivers}) given in the respective sections.

\section{Required tools}
\label{s:tools}

\section{Required materials}
\label{s:materials}

\section{Building the driver}
\label{s:driver}

\section{Building the enclosure}
\label{s:enclosure}

\section{Building the headband}
\label{s:headband}

\section{Building the earpads}
\label{s:pads}

\section{Assembly}
\label{s:assembly}

\section{Measurements}
\label{s:measurements}

\section{What to do different in the future}
\label{s:future}

\bibliographystyle{plain}
\bibliography{main}

\end{document}
